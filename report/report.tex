\documentclass[a4paper]{IEEEtran}
\begin{document}

\author{Catalin David}
\title{Channel Modelling Using Bellhop in Underwater Acoustic Sensor Networks}

\maketitle
\begin{abstract}
Abstract
\end{abstract}

\section{Introduction}
Studying and modelling how underwater sensor networks behave is a new and difficult task, but the rewards are incredible: exploration of the ocean which covers about 70\% of the earth surface -- this will ease the search of new energy sources, it will allow environmental and industrial monitoring (of animals and equipment), early disaster warning (for earthquakes, tsunamis) and, of course, for other military purposes.

While the design of such a system might seem related to the one of terrestrial networks, it is very different. Unlike the terrestrial counterpart, for underwater networks only acoustic communications are considered to be viable (at the physical layer) and this creates issues, since in the terrestrial setting you have radio frequency (RF) communications. In an underwater environment, RF has been tested and the results were not very good, the transmission suffering from severe atenuation, the only succesful deployment being at very low frequencies and with a large antenna and a very high transmission power. Still, this is not what one would have in mind when thinking about an underwater sensor network: an underwater sensor network consists of many interlinked sensors that are deployed once and with which there is no physical interaction for a long period of time. This generates many requirements, but the most important for the networking part are energy efficiency and a reliable transmission, leading to the study of acoustic channels.

From a networking point of view, the acoustic communication is very different from the RF communication, mainly due to the low bandwidth, propagation delay and the quality of the link. One other important thing to note is that underwater networks are, in general, very prone to noise due to the environment -- waves, shipping activity, a modification in the temperature or salinity of the water etc. We consider now a network with multiple nodes in which there might be multiple paths between each destination and receiver. If terrestrial protocols were to be used in such a volatile environment, when a link would be down, there would be an increased amount of traffic in the network for rerouting the transmission of data, leading to a high energy usage, ignoring one of the main requirements.

Since the development and deployment costs for such a network, alognside with the limited remote access to the deployed network,  one of the challenges involved is to develop a valid simulation model that is able to provide results that are according to real life. Once that is done, one can experiment with different theoretical implementations and find out more about the underwater communication constraints and limits.

\section{Background}

\subsection{Motivation}

\subsection{Previous Work}

\section{Methodology}

\section{Results}

\section{Future Work and Conclusion}

\begin{thebibliography}{99}
\bibitem{asd} dadafa
\end{thebibliography}

\end{document}
